\documentclass[a4paper,10pt,parskip]{scrartcl}
\usepackage[utf8]{inputenc}
\usepackage[ngerman]{babel}
\usepackage[T1]{fontenc}
\usepackage{lmodern}
\usepackage{amsmath}
\usepackage{amsfonts}
\usepackage{amssymb}
\usepackage{graphicx}
\usepackage{placeins}
\usepackage[lined,algoruled,linesnumbered]{algorithm2e}


\title{Treffen 07.11.2014}
\author{Dorle Osterode, Stefan Dang \& Lukas Götz}

\begin{document}

\maketitle{}

\section{Implementationsmöglichkeiten der benötigten Datentypen in Genometools}

Erste Ideen zur Implementierung der Datentypen fuer Scaffold-Graphen in C unter
Verwendung bestehender Datentypen aus Genome-Tools.

/* Scaffold-Graph */

typedef enum \{ VC\_UNIQUE, VC\_REPEAT,VC\_POLYMORPHIC, VC\_UNKNOWN \}
	 	VertexClass;\\
typedef enum \{ ED\_ANTISENSE, ED\_SENSE \} Direction;\\
typedef enum \{ EC\_REVERSE, EC\_SAME \} Composition;

/* Vertex */

struct GtScaffoldGraphVertex\\
\{\\
\begin{tabular}{l}
  /* eindeutige ID für den Knoten */\\
  GtUword id;\\
  /* Länge der Sequenz, die der Contig darstellt */\\
  GtUword seqlen;\\ 
  /* Wert der A-Statistik, um Contigs als REPEAT oder UNIQUE
     klassifizieren zu können;\\
     in Genom-Tools vom Typ float */\\
  float astat;\\	
  /* abgeschätzte Anzahl an Vorkommen des Contigs im Genom */\\
  float copynum;\\  
  /* zur Klassifikation des Knotens: REPEAT, UNIQUE, ... */\\
  VertexClass vertexclass;\\ 
  bool hasconflictinglink;\\
  GtUword nofedges;\\
  /* Sammlung von Kanten, die von dem Contig abgehen */\\
  struct GtScaffoldGraphEdge   **edges;\\  
  /* Markierung für Algorithmen; aus Genome-Tools entnommen siehe
  match/rdj-contigs-graph.c */\\
  GtContigsGraphMarks color;\\ 	  
\end{tabular}
\};

/* Edge */
struct GtScaffoldGraphEdge\\
\{\\
\begin{tabular}{l}
  /*  Knoten, zu dem die Kante führt */\\
  struct GtScaffoldGraphVertex *pend;\\ 
  /* Kante, die genau in die andere Richtung führt */\\
  struct GtScaffoldGraphEdge *ptwin;\\ 
  /* Informationen zu der Verbindung zwischen den Knoten */\\
  GtScaffoldGraphLink link;\\ 
  /* Markierung für Algorithmen */\\
  GtContigsGraphMarks color;\\ 
\end{tabular}
\};\\

/* Link */
struct GtScaffoldGraphLink\\
\{\\
\begin{tabular}{l}
  /* Id von dem Knoten auf den die Kante zeigt */\\
  GtUword id;\\ 	
  /* Länge der Sequenz des Contigs, auf den die Kante zeigt */\\
  GtUword seqlen;\\ 
  /*redundant ?*/\\

  /* Abschätzung der Entfernung der verbundenen Contigs */\\
  Gtword dist;\\
  /* Standardabweichung von der abgeschätzten Entfernung */\\
  float stddev;\\
  /* Anzahl der Distanzinformationen, die ein Anzeichen für die
  Verbindung der Contigs geben */\\
  GtUword numpairs;\\
  /* Typ der Verbindung: Distanz aus einer Eingabedatei\\
     (SLT-DISTANCEEST), (SLT-REFERENCE), (SLT-INFERRED),\\
     eine andere Verbindung (SLT-NOTYPE) */\\	    
  LinkType type;\\
  /* enthält die Richtung (Sense, Antisense) und welche\\
     Stränge die paired-Information enthalten (die gleiche\\
     Richtung oder das Reverse) */\\
  Direction direction;\\
  Composition composition;
\end{tabular}
\};

/* Graph */
struct GtScaffoldGraph\\
\{\\
\begin{tabular}{l}
  struct GtScaffoldGraphVertex **vertices;\\
  GtUword nofvertices;
\end{tabular}
\};

\section{Verfeinerung des Algorithmus zur Filterung der Knoten}

\section{Notizen zu dem Layout-Algorithmus}

\end{document}

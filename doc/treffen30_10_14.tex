\documentclass[a4paper,10pt,parskip]{scrartcl}
\usepackage[utf8]{inputenc}
\usepackage[ngerman]{babel}
\usepackage[T1]{fontenc}
\usepackage{lmodern}
\usepackage{amsmath}
\usepackage{amsfonts}
\usepackage{amssymb}
\usepackage{graphicx}
\usepackage{placeins}
\usepackage[lined,algoruled,linesnumbered]{algorithm2e}


\title{Treffen 30.10.2014}
\author{Dorle Osterode, Stefan Dang \& Lukas Götz}

\begin{document}

\maketitle{}

Anhand der verwendeten Datentypen in der Implementation des
Scaffoldings in dem Programm SGA, wurde eine Liste an grundlegend
benötigten Datentypen mit den jeweils benötigten Informationen
bzw. Feldern erstellt. Die Liste orientiert sich sehr stark an der
Implementation.

\begin{itemize}
\item Datentyp Graph: Der Graph enthält die Contigs als Knoten und
  Kanten zwischen zwei Knoten, wenn diese Knoten über eine
  \textit{paired-end} oder \textit{mate-pair} Information miteinander
  verbunden sind. Auf diesem Graph sollen dann die Algorithmen
  arbeiten. Der Graph ist bidirektional. Der Graph enthält folgende
  Informationen:
  \begin{itemize}
  \item Knoten
  \item Kanten
  \item ungewurzelt
  \end{itemize}
\item Datentyp Knoten: Ein Knoten ist spezialisiert und enthält alle
  Informationen über den zu repräsentierenden Contig direkt. Diese
  Informationen könnten allerdings auch in einen weiteren Datentyp
  ausgelagert werden. Der Vorteil einer Auslagerung wäre, dass eine
  generische Graph-Struktur und damit generische Graph-Algorithmen mit
  \textit{call-back}-Funktionen verwendet werden können, falls eine
  solche Graph-Struktur in den Genometools schon vorhanden ist. Ein
  Knoten bzw. Contig braucht folgende Informationen:
  \begin{itemize}
  \item VertexID m-id \# eindeutige ID für den Knoten
  \item size-t m-seqLen \# Länge des Sequenz, die der Contig darstellt
  \item double m-AStatistic \# Wert der A-Statistik, um Contigs als
    REPEAT oder UNIQUE klassifizieren zu können
  \item double m-estCopyNumber \# abgeschätzte Anzahl an Vorkommen des
    Contigs im Genom
  \item ScaffoldEdgePtrVector m-edges \# Sammlung von Kanten, die von
    dem Contig abgehen
  \item ScaffoldVertexClassification m-classification \#
    Klassifikation: REPEAT oder UNIQUE
  \item GraphColor m-color \# Markierung für Algorithmen
  \item bool m-hasConflictingLink \# $\top$, wenn der Contig im
    Konflikt stehende Kanten hat
  \end{itemize}
\item Datentyp Kante: Eine Kante verbindet zwei Knoten und hat eine
  Richtung. Eine Kante braucht folgende Informationen:
  \begin{itemize}
  \item ScaffoldVertex* m-pEnd \# der Knoten, zu dem die Kante führt
  \item ScaffoldEdge* m-pTwin \# Kante, die genau in die andere
    Richtung führt
  \item ScaffoldLink m-link \# Informationen zu der Verbindung
    zwischen den Knoten
  \item GraphColor m-color \# Markierung für Algorithmen
  \end{itemize}
\item Datentyp Link: Ein Link enthält alle Informationen zu einer
  Verbindung zwischen zwei Knoten. Ein Link hat folgende Information
  über eine Kante:
  \begin{itemize}
  \item std::string endpointID \# Id von dem Knoten auf den die Kante zeigt
  \item int seqLen; \# Länge der Sequenz des Contigs, auf den die Kante zeigt
  \item ScaffoldLinkType type \# Typ der Verbindung: Distanz aus einer
    Eingabedatei (SLT-DISTANCEEST), (SLT-REFERENCE), (SLT-INFERRED), eine
    andere Verbindung (SLT-NOTYPE)
  \item EdgeData edgeData \# enthält die Richtung (Sense, Antisense)
    und welche Stränge die paired-Information enthalten (die gleiche
    Richtung oder das Reverse)
  \end{itemize}
  In der Implementation von SGA hat ein Link noch folgende
  Informationen, die allerdings nicht weiter erklärt wurden. Diese
  Informationen werden aus den Eingabedateien übernommen:
  \begin{itemize}
  \item int distance
  \item double stdDev
  \item int numPairs
  \end{itemize}
\end{itemize}



\begin{algorithm}[H]
  \SetAlgoLined
  \KwData{Contigs, Distanzinformationen}
  \KwResult{Scaffold der einzelnen Contigs}
  Graph konstruieren mit Contigs als Knoten und Kanten aus den Distanzinformationen\;
  repititive Knoten anhand von Astatistik entfernen\;
  \dots alle möglichen Dinge herausfiltern, wenn gewünscht (nicht default)\;
  Polymorphe Knoten entfernen und inkonsistente Kanten löschen\;
  Anfangs- und Endknoten von Zyklen löschen\;
  Scaffold aus dem gefilterten Graphen berechnen\;
  \caption{Grober Ablauf des Scaffoldings bei SGA}
\end{algorithm}

\end{document}

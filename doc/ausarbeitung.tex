\documentclass[a4paper,10pt,parskip]{scrartcl}
\usepackage[utf8]{inputenc}
\usepackage[ngerman]{babel}
\usepackage[T1]{fontenc}
\usepackage{lmodern}
\usepackage{amsmath}
\usepackage{amsfonts}
\usepackage{amssymb}
\usepackage{amsthm}
\usepackage{siunitx}
\usepackage{epsfig}
\usepackage{tikz}
\usepackage{algpseudocode}
\usepackage{algorithm}
\usepackage{graphicx}
\usepackage{placeins}

\usetikzlibrary{shapes.geometric, calc}

\title{Projektausarbeitung \\\vspace{.5cm} \large gt Scaffolder: Eine Scaffolding-Software}

\author{Dorle Osterode, Lukas Götz \& Stefan Dang}
\date{}

\begin{document}
\maketitle{}
\thispagestyle{empty}
\begin{abstract}
  TODO: wollen wir ein abstract haben?
\end{abstract}

\newpage{}
\setcounter{page}{1}
\section{Einleitung}
TODO: was ist DAS scaffolding Problem?

Nach der Assemblierung von Sequenzen ist das Zielgenom oftmals noch
nicht vollständig rekonstruiert. Die rekonstruierten Sequenzabschnitte
(Contigs) sind dabei unabhängig voneinander und die relative Anordnung
sowie die Orientierung sind unbekannt. Diese Lücken (Gaps) in der
Zielsequenz entstehen oft durch ungleichmäßige Abdeckung durch die
Sequenzierung sowie repetitive Teilsequenzen. Durch repetitive
Teilsequenzen können die Contigs nicht in eine eindeutige Reihenfolge
gebracht werden.

Durch das Scaffolding nach der Assemblierung sollen die Contigs
relativ zueinander angeordnet werden, sodass zusätzlich auch die
Orientierung stimmt. Um dieses Verfahren anwenden zu können müssen
Distanzinformationen zwischen den Contigs vorhanden sein. Diese
Informationen stammen normalerweise aus \textit{paired-end}- oder
\textit{mate-pair}-Sequenzierung.

Beim Scaffolding wird ein Graph zur Repräsentation der Beziehungen
zwischen den Contigs verwendet. Dieser Graph enthält für jeden Contig
einen Knoten und es gibt Kanten zwischen zwei Knoten, wenn es ein
Read-Paar gibt, dass die durch die Knoten repräsentierten Contigs
verbindet. Die Kanten enthält dabei Informationen über die Distanz und
die Orientierung.

Nach Huson (2002) (TODO: Zitat) ist das Scaffolding Problem
NP-vollständig. Dabei ist das Scaffolding die Ermittlung des optimalen
Pfades (Pfad mit maximalem Kantengewicht) jeder
Zusammenhangskomponente des Scaffolding-Graphen. Um diese Komplexität
zu umgehen werden Heuristiken zur Bestimmung eines guten Pfades für
jede Zusammenhangskomponente verwendet.

\section{Ziel des Projektes und Arbeitsvorgehen}
\subsection{Ziel des Projektes}
Ziel des Projektes war die Implementierung und Evaluierung einer
Scaffolding-Software. Dabei sollten die in der GenomeTools-Bibliothek
(TODO: referenz) vorhandenen Konzepte und Infrastruktur verwendet
werden. Zusätzlich sollte die Scaffolding-Software mit den Ausgaben
des Assemblers ReadJoiner kompatibel sein.

Um diese Ziele zu erreichen wurde der Scaffolder aus der SGA-Pipeline
(TODO: referenz) als Vorlage verwendet. Dieses Programm wurde
ausgewählt, da es sehr gute Ergebnisse erzielt (TODO: referenz des
benchmarking-papers). Da die SGA-Pipeline allerdings viele
Abhängigkeiten hat und deshalb nicht einfach zu installieren ist und
zusätzlich die Methodik des Scaffoldings nicht hinreichend
dokumentiert ist, sollte die Methodik aufgeklärt und der
Scaffolding-Algorithmus reimplementiert werden.

\subsection{Arbeitsvorgehen}
Zuerst wurden die verwendeten Datenformate und die Algorithmen
aufgeklärt. Da diese nicht hinreichend dokumentiert waren, mussten
diese Informationen aus dem Source-Code extrahiert werden.

Die so erhaltene Vorlage wurde leicht modifiziert, um sowohl Laufzeit
als auch Speicherplatz einzusparen. Unklarheiten in der Methodik wurden
dabei weitestgehend erst übernommen, um die Ergebnisse reproduzieren
zu können.

Diese Algorithmen wurden in der Programmiersprache C
implementiert. Dabei wurde gt Scaffolder zuerst so designet, dass es
die gleichen Eingabedateien erforderte wie SGA-Scaffold und die
gleichen Ausgaben produzierte, damit die Ergebnisse verglichen werden
konnten.

Um die Implementation zu testen wurden verschiedene Testdaten ausgesucht. (TODO!!!!)

Anhand der Testdaten wurde gt Scaffolder evaluiert. 

\section{Methoden}

\section{Ergebnisse}

\section{Diskussion und Ausblick}

\end{document}
